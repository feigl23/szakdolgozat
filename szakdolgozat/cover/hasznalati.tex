\pagestyle{empty}

\noindent \textbf{\Large A CD melléklet tartalma}

\vskip 1cm

\noindent A dolgozathoz tartozó melléklet a következőket tartalmazza.

\begin{itemize}
\item \texttt{dolgozat.pdf}: a dolgozat PDF fájl formájában,
\item \texttt{szakdolgozat}: a dolgozat \LaTeX\ forráskódját tartalmazó jegyzék,
\item \texttt{programok}: az elkészített programokat tartalmazó jegyzék.
\end{itemize}

\bigskip

\noindent A programok teszteléséhez a következő hardver és szoftver eszközökre van szükség.
\begin{itemize}
	\item Kamera,
	\item egy, a megfelelő könyvtárba tartozó ArUco marker,
	\item \textit{OpenCV} és \textit{ArUco} függvénykönyvtárak a kamerapozíció becsléséhez,
	\item \textit{PyOpenGL} és \textit{PyGame} függvénykönyvtárak a megjelenítéshez,
	\item \texttt{falcon}, \texttt{waitress} és \texttt{requests} csomagok a szerver és kliens közötti kommunikációhoz.
\end{itemize}

\bigskip

\noindent \textbf{A Python környezet és a függőségek telepítése}

\medskip

A Python értelmező és futtatókörnyezet telepítése platformonként változó.
\begin{itemize}
	\item Windows rendszerek esetében a Python hivatalos weboldaláról célszerű beszerezni a telepítőket: \url{https://www.python.org/downloads/}.
	\item Linux rendszereken a Python általában a rendszer része, vagy a csomagkezelővel telepíthető.
\end{itemize}

\medskip

A függőségeket érdemes külön virtuális környezetbe telepíteni.
Virtuális környezetet létrehozni az alábbi paranccsal lehet.
\begin{python}
python -m venv .venv
\end{python}
A környezetet ezt követően aktiválni kell az \texttt{activate} futtatásával:
\begin{python}
source .venv/bin/activate
\end{python}
(Windows rendszer alatt ez az \texttt{activate.bat} segítségével oldható meg.)

\noindent Ezt követően a függőségek az alábbi parancs kiadásával telepíthetők:
\begin{python}
pip install opencv-python aruco PyOpenGL pygame falcon requests waitress
\end{python}
A programok futtatásához a megfelelő Python fájlt kell megadni a Python értelmezőnek, például:
\begin{python}
python game.py
\end{python}
