\Chapter{Kiterjesztett és virtuális valóság}

\begin{itemize}
\item[AR] (\textit{Augmented Reality}): A kiterjesztett- vagy augmentált valóság, a valóság egy kiterjesztése, bővítése. 
\item[VR] (\textit{Virtual reality}): A virtuális valóság, a valóság teljes kizárása és virtuális környezetbe kerülés.
\item[MR] (\textit{Mixed Reality}): az AR és VR keveréke, a valós környezetre épít, ahogy az AR, azonban itt a cél a virtuális és valós környezet határának elmosása, nagyobb hangsúlyt kap az interakció. 
\end{itemize}

\Section{Hardware}

\SubSection{AR hardware}

A kiterjesztett valóság kipróbálásához elegendő lehet egy újabb verziójú Android (7.0-tól felfelé) vagy iOS (11-től felfelé) operációs rendszerrel ellátott okoseszköz is (pl. mobiltelefon, tablet).
Számos valóság kiterjesztő alkalmazás, játék érhető el ezen platformokra. 
Ezen eszközök a kiterjesztett valósághoz nélkülözhetetlen kamerával és olyan szenzorokkal vannak ellátva, amik szükségesek a AR alkalmazások használatához.
Az érzékelők mikro-elektromechanikai (MEMS, \textit{MicroElectroMechanical Systems}) 
rendszerek, azaz olyan apró ($1-100\mu\ m$ között) rendszerek amelyek mechanikai és elektronikai alkatrészeket is tartalmaznak.
\begin{itemize}
\item Inerciális szenzor: accelerometer és giroszkóp.
\item Gyorsulásérzékelő (accelerometer) : a három tengely irányába méri a gyorsulását az eszköznek. (pl. gravitáció). 
\item Giroszkóp szenzor: szögelfordulás mérő. (harmonikus rezgőmozgás).
\item Fontos érzékelő még a távolságérzékelő, iránytű, fényváltozás érzékelő.
\end{itemize}

Nem csak az okos eszközök, hanem a személyi számítógépek kameráját felhasználva is lehetőségünk van a kiterjesztett valóság tesztelésére.  A Unity legújabb verziói (2019.3-tól) már nem csak a Windowst támogatják, hanem hivatalosan, teljes támogatással lehetővé teszik a Linux operációs rendszerekre való fejlesztést.

Emellett számos speciális, kifejezetten erre a célra fejlesztett eszköz elérhető, 2019 legnépszerűbb darabjai a következők:
\begin{itemize}
\item Microsoft Hololens 2 :  egy AR headset, átlátszó lencsékkel. Számos innovációja közé tartozik az írisz felismerés, szemmozgás követés. Könnyű, jól vezérelhető.
\item Magic Leap One:  AR headset, amit egy ‘Lightpack” nevű, apró, övre csatolható számítógép működtet. 
\item Epson Moverio: AR headset, átlátszó lencsékkel. Létezik iparra koncentráló változata is, mert a kialakítása megfelel az olyan helyekre, ahol kötelező a védősisak használata. Robosztus kialakítása van. 
\item Google Glass: hangvezérléssel működik, így nincs szükség kézi irányításra. Android operációs rendszer az alapja, csatlakozik az internethez.
\end{itemize}

\SubSection{VR hardware}

A virtuális környezet megteremtéséhez szükség van egy szenzorrendszere, amely meghatározza a használója fej, test, kéz mozgását, továbbá egy olyan eszközre, amely az erő és nyomás érzékelésre alkalmas.

A valóság kizárásához szükséges még egy audiorendszer, amely a tér hangjait hivatott adni és egy megjelenítő rendszerre, amely magát a teret vetíti a használó elé (például VR szemüveg).
Mindehhez nélkülözhetetlen a megfelelő, VR alkalmazást futtatni képes számítógép, okoseszköz, játékkonzol. 

Az első VR szemüveg Oculus Rift volt, meglátva benne a lehetőséget a nagy cégek saját VR szemüveggel álltak elő ilyen például az Playstation VR, a Samsung Gear VR, a Sony Project Morpheus, HTC Vive.

Ezek közül a Samsung Gear VR Samsung okostelefonnal, a Playstation VR és a Sony Playstation konzollal, az Oculus Rift és a HTC Vive pedig számítógéppel kompatibilis. A legnépszerűbbnek és legjobb élményt nyújtónak még mindig az Oculus Rift számít.

A piacon a szemüvegek mellett számos kontroller is elérhető, amik segítségével különböző inputokat vihetünk be, a mozgást irányíthatjuk, a kéz mozgását és a lenyomott gombokat figyelembe véve. (Ilyen kontroller pl. HTC Vive.) A 3DRudder kialakítása pedig képessé teszi a használóját, hogy a lába segítségével irányítson, elérhető továbbá konzol kontroller mintájára készített VR kontroller (Steelseries stratus XL) és 3D egér is (3DConnexion).

A kontroller mellett VR kesztyűt is igen népszerű, ugyanis a kéz mozgását pontosabban alakítják digitális utasítássá.  A szenzorok beépítettek, külön követik az ujjak mozgását, mérik a gyorsulást, térbeli orientációt és gravitációs erőt.

Glove One-nal súlyt is lehet szimulálni, a Teslasuit Glove, a HaptX lehetővé teszi, hogy a használója érezze a virtuális térben megfogott objektumot. 
A Teslasuit-nak van továbbá VR ruhája is, ami segítségével még könnyebb és pontosabb a test mozgásának a követése, valamint képes a hőmérsékletet változteni lokálisan, így meleg vagy hideg hatást kelt a használója bőrén.

\Section{AR típusok}

\SubSection{Marker alapú (marker based)}

Marker:  egy speciális azonosító, melyet  felismernek az érzékelők és a virtuális tartalmat (pl. 3D objektum) a relatív pozícióhoz igazítják.
Marker lehet QR kód vagy egyszerűbb szimbólum, fekete négyzet, de adott esetben kép vagy objektum is. A lényeg, hogy egyedi legyen a környezetben és jól felismerhető.
Ha az eszköz elmozdul, a marker eltűnik vagy nem felismerhető a tartalom eltűnik.

\SubSection{Marker nélküli (markerless)}

A SLAM (Simultaneous Localization and Mapping) az alapja. 
Feltérképezi a környezetet és 3D modellként értelmezi, a tartalmat ehhez viszonyítva lehet pozícionálni, tájolni. A virtuális tartalom itt nem tűnik elmozdulás miatt, hisz nem egy adott markerhez, hanem az egész környezethez viszonyít.

\SubSection{Projeciós}

A legegyszerűbb típus. Fény vetítése (virtuális tartalom) a valós felülete és interakció ezzel. (pl numpad vetítése és a számok megérintése)

\SubSection{Superimposition}

Alternatív képet kapunk egy valós objektumról azáltal, hogy a valós látványt kicseréljük egy kiterjesztett-tel. (pl. egy emberre irányítva a kamerát a látvány kiegészül a csontjaival.) Használják a katonaságnál, egészségügyben. Ez már inkább MR. 

\Section{Software, SDK}

\SubSection{Apple ARKit}

iOS 11-től elérhető, iPhone-ra és iPad-re való AR alkalmazások fejlesztéséhez. Visual-Inertial odometry (VIO) használ az eszköz mozgásának érzékeléséhez, meghatároz vízszintes objektumokat, mint például: padló, asztal. Az újabb verzió már jobban bevonja az embert is, az emberi test mozgását, pozícióját is meghatározza.

\SubSection{Google ARCore}

Androidos eszközökre (7. Verziótól felfelé).
Mozgáskövetés, fényérzékelés, környezet felmérése (függőleges és vízszintes tárgyak felmérése, méretük)

\SubSection{Vuforia}

A 2D-s és 3D-s objektumok valós időben történő felismerésére, nyomon követésére. Az eszköz kameráján át tekintve lehetőség van az objektum pozícionálásra a valós objektumokhoz viszonyítva.

Számos marker típust támogat, a “marker nélküli” kép- és objektum targeteket is, továbbá a VuMarkot.
Elérhetők API-k: C++, Java, .NET nyelven a Unity révén, így iOS, Android és Windows és Linux operációs rendszerekkel ellátott eszközökre is fejleszthetők Vuforiát használva alkalmazások.

\SubSection{Wikitude}

Helyzetmeghatározást végez ( ez volt az első) a felhasználó pozícióját (Wifi vagy GPS) és irányát figyelembe véve. (accelerometer és iránytű segítségével)
Ezenkívül képfelismerést és SLAM (simultaneous localization and mapping) használ. Az utóbbi lehetővé teszi, hogy elhagyjuk a markereket.
iOS, Android, Windows.

\SubSection{EasyAR}

Marker alapú, kiváló kezdőknek. Lehetőség van Androidra és iOS-re, Windowsra való fejlesztésre is.

\SubSection{Kudan}

Támogatja a marker és a marker nélküle használatot is. 
Lehetőség van Androidra és iOS-re, Windows-ra való fejlesztésre is.
Tapasztaltabb fejlesztőknek ajánlott. 

\SubSection{ARToolKit}

Nyílt forráskódú. (GitHub-on megtalálható).
Fekete négyzetet használ markerként.
ARToolKit jelenlegi verziója: Windows, Mac OS,  Linux, Android, iOS
Unity plugin-ként: Androidra és iOS

\SubSection{MaxST}

Kép követés, SLAM, objekt követés, QR/Barcode szkenner.

\SubSection{AR Lab}

Könnyű használni. A keretrendszer gondoskodik a funkciókról, leginkább a kinézetet kell alakítani.
iOS, Android.

\SubSection{Pikkart AR SDK}

Könnyű használni, gyors.
iOS, Android

\SubSection{További API-k}

\begin{itemize}
\item Banuba FaceAR
\item Google Cardboard
\item WebXR Device 
\item Vuforia VuMArk Generation
\item NGrain
\end{itemize}

\Section{Unity}

A Unity egy multiplatform játék motor (game engine), amellyel lehetőségünk van Windowsra Linuxra, Mac OS-re, iOS-re, Androidra, valamint Wii-re és konzolokra fejleszteni.

Van üzleti és személyi változata is. Bizonyos keretek alatt ingyenes lehet a cégeknek is, nem csak a magánszemélyeknek.
