\Chapter{Összefoglalás}

A dolgozat áttekintette az elterjedt AR/VR eszközöket, azok funkcióit. Ez ahhoz volt szükséges, hogy adjon egy rálátást, hogy a témakörben milyen lehetőségek vannak a virtuális terek kialakítására.

A bemutatott rendszer egy nagyon lényeges eleme az, amelyik a kameraképből megállapítja a kamera pozícióját a virtuális térben. Ennek technikai megvalósítása egy külön fejezetet kapott. Lényegi kérdés volt, hogy szükség van-e markerre a probléma megoldásához. Miután a vizsgálatok alapján kiderült, hogy a pozícióbecslés csak azok segítségével tud megbízható lenni, a megfelelő marker típus kiválasztása jelentette a következő megoldandó feladatot. A fellelhető leírások és szoftveres eszközök alapján az ArUco markerre esett a választás.
Külön figyelmet igényelt a kamera kalibrálása, a marker paramétereinek meghatározása, majd az összeállított pozícióbecslő rendszer tesztelése.

A dolgozat következő nagyobb egysége a tér megjelenítésével foglalkozik.
A megfelelő technológiák megválasztása ebben az esetben sem volt egyszerű.
Az OpenCV-vel és OpenGL-el való egyszerű integrálhatóság alapján a Python programozási nyelv került kiválasztásra.

Össze kellett gyűjteni és elő kellett készíteni a virtuális tér objektumait.
Ezek esetében is külön kisebb, egymástól függetlenül futtatható alkalmazások készültek, amelyekkel ki lehet próbálni a rendszer funkcióit.

Definiálásra majd megvalósításra került az a szabályrendszer, amely megszabja a virtuális térben a kooperáció módját.

A dolgozat ezt követően leírja az eszközök összekapcsolásának REST API-n keresztül történő megvalósításának módját koncepció szintjén.

Összességében elmondható, hogy a dolgozat jellegükben erősen eltérő problémák megoldásait és azok integrálásának módját mutatja be.

Későbbi fejlesztési lehetőségként többek között új modellek és animációk hozzáadása, teljesítményoptimalizálás és a rendszer kooperatív feladványokkal történő bővítése szerepel.
