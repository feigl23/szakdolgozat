\Chapter{Bevezetés}

A kiterjesztett valóság (AR, \textit{Augmented Reality}) és a virtuális valóság (VR, \textit{Virtual Reality}) eszközkészlete folyamatosan bővül.
A néhány évtizede még laboratóriumi körülmények között tesztelt megjelenítők és érzékelők jelentős része már kereskedelmi forgalomban is beszerezhető.
Az AR alkalmazások egy részét okostelefon segítségével is ki lehet próbálni.
Ezek alapján kijelenthető, hogy a technológia már kifejezetten elérhető stádiumban van.

A dolgozat az AR témakörön belül kooperatív tér kialakításával foglalkozik.
Ez azt jelenti, hogy egy olyan alkalmazás elkészítését mutatja be, amelyik egyidejűleg több eszköz számára képes ugyanazt a virtuális teret megjeleníteni különböző nézőpontokból, abban a felhasználók interakciókat tudnak végrehajtani.

A dolgozat először áttekinti a kiterjesztett és a virtuális valóság elemeit.
Az alapfogalmak tisztázását követően felsorol néhány elterjedt hardveres eszközt, azok fő funkcióit.

A következő vizsgált problémakör a kiterjesztett valóság szoftveres megvalósításával foglalkozik, feltételezve, hogy a bemenet az egy kamerakép.
Ezzel kapcsolatban a kamerapozíció becslése egy külön fejezetben kerül tárgyalásra.
Ennél a megoldandó feladatot a megfelelő marker kiválasztása, és az alapján a kamerapozíciók becslése jelenti.

A virtuális kollaborációs tér a 4. fejezetben részletesen bemutatásra kerül. Ez magába foglalja a tér megjelenítését, objektumainak mozgatását és az ehhez szükséges irányítás megvalósítását.
A dolgozatban példaként bemutatott virtuális tér egy kastély, melyben a felhasználók pingvineket tudnak irányítani.
A játék egy kollaboratív módon megoldandó feladatot ad a felhasználóknak, aminek a célja, hogy a térben elhelyezett dobozok megfelelő mozgatásával ki tudjanak jutni a kastélyból.

A virtuális térben az eszközök közötti kommunikációnak a módját vizsgálja az 5. fejezet. Ebben leegyszerűsített példák segítségével láthatjuk, hogy a kliens és a szerver között milyen elérési pontokon keresztül milyen adatok küldésére kerül sor.

A 6. fejezet összegzi az elkészített programokat. Ebben először felsorolásra kerülnek azok a demó programok, amelyek az egyes funkciók elkészítéséhez, teszteléséhez szükségesek voltak, majd az ezek segítségével összeállított alkalmazás bemutatására kerül sor.
