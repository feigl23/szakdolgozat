\Chapter{Interakció és kommunikáció a virtuális térben}

Ahhoz, hogy a kliensek kooperálni tudjanak a virtuális térben meg kell oldani az eszközök közötti kommunikációt.
Feltételezhetjük, hogy az eddigiekben bemutatott alkalmazásnak (továbbiakban kliensnek) van hozzáférése a helyi hálózathoz.
Ezáltal szerver-kliens architektúra szerint megoldható az eszközök közötti kommunikáció.

A következő szakaszokban azt vizsgáljuk meg, hogy a központi szerver milyen feladatokat lát ez, azzal a kommunikáció hogyan oldható meg interfészek, protokollok szintjén.

Az ezt követő részben a korábban bemutatott kliens alkalmazás kiegészítése kerül részletezésre, hogy az alkalmas legyen arra, hogy a többi klienssel megossza az adatokat a központi szerveren keresztül.

\Section{Központi szerver}

Egyenrangú kliens alkalmazásokat feltételezve a virtuális térben lévő ágensek és entitások adatait a központi szervernek kell kezelnie.
A szerverben tehát megjelenik a kliensben korábban bemutatott világmodell egy része.

Az adatmegosztási probléma tulajdonképpen úgy is tekinthető, hogy a kliensek szinkronizálják a szerverben lévő adatmodellt.
A szinkronizálásnak itt két fő iránya van.
\begin{itemize}
	\item A kliensnek le kell tudni kérdezni, hogy aktuálisan a térben hol vannak az entitások.
	\item A saját pozícióját vissza kell tudnia küldeni a szervernek.
\end{itemize}
Az elemek azonosításához szükséges, hogy minden ágensnek egyedi azonosítója legyen.

További tényező, hogy a játéknak a kezdetekor tudni kell, hogy mennyi ágens lesz majd, mivel az szerepet játszik a további elemek létrehozásában és elhelyezésében.
Egyszerűbb esetben azt mondhatjuk, hogy a szerver indításakor megadjuk, hogy mennyi ágens szerepeljen az adott térképen, majd akkor fog ténylegesen létrejönni a kooperációs tér, amikor az adott számú résztvevő már bejelentkezett.
Ezt valamilyen formában jelezni kell a kliensek felé is.
Egy lehetséges megoldás, hogy a virtuális tér állapotára vonatkozóan ilyen esetben egy speciális üzenetet kap, amelyben egyúttal az is szerepel, hogy mennyi ágens van már bejelentkezve és összesen mennyire van szükség.

% TODO: Tulajdonképpen a World adatszerkezetet kell megosztania.

% TODO: Feltételezzük, hogy egyetlen közös virtuális tér van.

\Section{REST API}

A REST alapvetően egy állapotmentes protokoll.
Az a fő előnye, hogy az erőforrások felé intézett kérésekben minden olyan információ szerepel, amely a válasz kiszámításához/összerakásához szükséges \cite{sohan2017study}.

Tegyük fel, hogy a kooperációs teret egy \texttt{world} nevű elérési pont jelöli.
Az elérési útvonalban ezt érdemes úgy szerepeltetni, hogy \texttt{/api/world}.
Elsődlegesen ez felelős a virtuális kollaborációs tér adatainak a megosztásáért.
E mellett még további erőforrásokra is szükség van, amelyeket funkciók szerint csoportosítva részletez a dolgozat a következő szakaszokban.

\SubSection{Bejelentkezés a virtuális térbe}

A kooperációs térhez való csatlakozáshoz authentikálni kell a klienst.
Mivel nem egy szenzitív alkalmazásról van szó, ezért ez esetben a kliensnek elegendő átküldenie a nevét a szervernek az \texttt{/api/login} elérési pontra.
Egy példa bejelentkezési kérés a szerver felé.
\begin{python}
{ "name": "Agent Name" }
\end{python}
Erre a szerver (sikeres bejelentkezés esetén) az ágens saját, egyedi azonosítójával tér vissza.
Például, hogy ha az első bejelentkező ágensről van szó, akkor
\begin{python}
{ "agent_id": 1 }
\end{python}
Amennyiben valamilyen probléma lépne fel a kérés kiszolgálása közben, akkor a szerver hibaüzenetet ad vissza (HTTP hibakód formájában).

\SubSection{Entitások adatainak lekérdezése}

Az adatok lekérdezéséhez a kliens egy GET kérést küld a \texttt{world} erőforrás felé.
Az a válaszüzenetben visszaadja az összes entitás állapotát.

Tegyük fel, hogy a térben két ágens és két mozgatható elem van csak.
Az egyik ágens le szeretné kérdezni, hogy melyik entitás hol van a térben.
Ehhez egy GET kérést kell küldenie a kliensnek a \texttt{/api/world} elérési ponthoz.
Alapesetben ebben nem kell szerepelnie semmilyen további adatnak, mivel feltételezzük, hogy a kliens számára az összes adatra szüksége van.
A kérésre a választ az alábbi formában kapja.
\begin{python}
{
  "entities": [
    {
      "id": 1,
      "position": [1, 2, 3],
      "direction": [2, -3, 0]
    },
    {
      "id": 2,
      "position": [5, 1, 4],
      "direction": [2, 1, 0]
    },
    {
      "id": 3,
      "position": [3, 1, 4],
      "direction": [-1, -2, 0]
    },
    {
      "id": 4,
      "position": [5, 0, 7],
      "direction": [3, 3, 0]
    }
  ]
}
\end{python}
Ebben tehát az entitások egyedi azonosítói szerint szerepelnek a hozzájuk tartozó pozíciók és irányok.

Amennyiben a kliens még nem lenne bejelentkezve, úgy erre a kérésre egy üres válasz fog érkezni (tehát hogy a kliens számára elérhető aktuális virtuális térben nincsenek még elemek).

\SubSection{Saját pozíció visszaküldése}

Az irányított ágens adatainak küldéséhez a klienseknek egy POST kérést kell intéznie a szerver felé.
Definiáljunk egy \texttt{agent} nevű erőforrást ehhez.
A kérésből egyértelműen ki kell derülnie, hogy az melyik ágensre vonatkozik.
Erre egy kézenfekvő megoldás, hogy ha magában az üzenetben szerepel az ágens egyedi azonosítója is, amelyet az ágens a virtuális térbe való bejelentkezéskor kap a szervertől.

% TODO: POST -> Saját pozíció felterjesztése, az ágens adatait kell szerializálni.

\begin{python}
{
  "agent_id": 1,
  "x": 10, "y": 20, "z": 30,
  "rotation": 40
}
\end{python}

\Section{Kliens alkalmazás}

A kliens alkalmazásnak a központi szerver erőforrásaihoz kell hozzáférnie.
Ilyen jellegű alkalmazásoknál tipikusan a kliens az egy webböngésző, jelen esetben viszont egy grafikus Python alkalmazásról van szó.
A kérések kezelését ilyen esetben például a \texttt{requests} modul segítségével lehet egyszerűen megoldani.

% https://www.w3schools.com/python/module_requests.asp

Feltételezhetjük, hogy a szerver szintén az alhálózaton van, továbbá, hogy az üzenetek nem túl nagy méretűek, így a kérések küldése és feldolgozása szinkron módon történhet.

\SubSection{Pozíciók interpolálása}

A rendszer szempontjából a központi szerveres megvalósítás szűk keresztmetszetet jelenthet, olyan tekintetben, hogy nem reális, hogy minden kliens minden képkocka kirajzolás előtt lekérdezi a szerveren lévő aktuális adatokat.
Tegyük fel ugyanis, hogy minden kliens 25 képkockát rajzoltat ki, és egyidejűleg 10 kliensünk van.
Hogy ha minden kliens egyszer lekérdezi az aktuális állapotot, és átküldi a saját adatait, akkor az azt jelenti, hogy másodpercenként 500 kérést kellene feldolgoznia a szervernek.
A feladat jellegéből adódik, hogy az ágensek számával a kérések száma lineárisan fog változni.
A konstans tényezőn (tehát a szükséges lekérdezések számán kliensenként) viszont lehet javítani azzal, hogy ha például a szinkronizációs pontok alacsonyabb frekvenciára vannak beállítva.
Ennek az optimális értéke függ a kliensek számától, a hálózati kommunikáció sebességétől, továbbá a szerver saját válaszidejétől.

\Section{Megbízhatóság és biztonság}

Mivel az alkalmazás több számítógépen működik egyidejűleg, ahol a kliens egy eseményvezérelt program, így több tekintetben is egy aszinkron alkalmazásról beszélhetünk.
A következő szakaszokban felsorolásra kerül néhány probléma, amely felmerült a rendszer kialakítása közben.

\SubSection{Időben eltérő világállapot}

A konstrukcióból adódik, hogy a kooperatív tér aktuális állapota a rendszer minden eleménél (beleértve az összes klienst és a központi szervert is) eltérő.
Ez abból adódik, hogy az állapot szinkronizálása időbe telik, és bármelyik kliensben történhet olyan változás (például billentyűesemény) amelyről a szerver (és így a többi kliens sem) értesült még.
Ilyen szempontból tehát a szerver nem jelent kivételt, mivel az az adatokat a kliensektől kapja eltérő időpontokban.

\SubSection{Kliensek jogosultságainak kezelése}

A példaként létrehozott kollaborációs tér egy egyszerű példája annak, hogy hogyan lehet több klienst egy virtuális térbe összekapcsolni, és azt különböző perspektívákból megjeleníteni.
Nagyobb, komplikáltabb alkalmazások, terek és feladatok esetében érdemes a térhez való hozzáférést szigorítani, azt hasonló jogosultságokhoz kötni, mint amelyek úgy általában a webalkalmazásokban megtalálhatóak.
(A kooperációs térben való részvétel tehát a hagyományos értelemben vett \textit{session}-ként kezelhető.)

\SubSection{Kliensek és a szerver kiesése}

Különböző okok miatt (például hálózati vagy eszköz problémából adódóan) megszakadhat a kapcsolat a szerver és a kliens között.
Ez többféle formában is jelentkezhet.
\begin{itemize}
	\item Rövid, átmeneti hálózati hibáról van szó. A klienseknek (vagy akár a szervernek is) érdemes interpolálni a kimaradt időszeletet.
	\item Egy kliens teljesen kieshet (gyakorlatilag, mint ha kijelentkezne).
	\item Amennyiben például a hálózat hiba vagy a megnövekedett feldolgozási idő csak késleltetést okoz, akkor érdemes lehet a használhatatlan üzeneteket (állapotfrissítéseket) eldobni.
	\item A szerver hosszabb kiesése esetén külön hibaüzenet formájában érdemes jelezni a hibát a klienseknek.
\end{itemize}

Az említett problémák egy kis részét képezik az aktuálisan bemutatott alkalmazáshoz hasonló aszinkron, elosztott "valós" idejű alkalmazásoknál.
Az alkalmazás későbbi fejlesztése során érdemes ezeket és a lehetségi megoldásokat közelebbről megvizsgálni.
