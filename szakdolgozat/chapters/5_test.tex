\Chapter{Interakció és kommunikáció a virtuális térben}

Ahhoz, hogy a kliensek kooperálni tudjanak a virtuális térben meg kell oldani az eszközök közötti kommunikációt.
Feltételezhetjük, hogy az eddigiekben bemutatott alkalmazásnak (továbbiakban kliensnek) van hozzáférése a helyi hálózathoz.
Ezáltal szerver-kliens architektúra szerint megoldható az eszközök közötti kommunikáció.

A következő szakaszokban azt vizsgáljuk meg, hogy a központi szerver milyen feladatokat lát ez, azzal a kommunikáció hogyan oldható meg interfészek, protokollok szintjén.

Az ezt követő részben a korábban bemutatott kliens alkalmazás kiegészítése kerül részletezésre, hogy az alkalmas legyen arra, hogy a többi klienssel megossza az adatokat a központi szerveren keresztül.

\Section{Központi szerver}

Egyenrangú kliens alkalmazásokat feltételezve a virtuális térben lévő ágensek és entitások adatait a központi szervernek kell kezelnie.

% TODO: Kezelt adatok, adatszerkezetek

\Section{REST API}

% TODO: GET -> összes pozíció lekérdezése
% TODO: POST -> Saját pozíció felterjesztése

\Section{Kliens alkalmazás}
