\Chapter{Kamerapozíció becslése a virtuális térben}

\Section{Markerek}

\SubSection{Klasszikus markeres megoldások}

Vonalkód, QR-kód, Aruco marker
% kellenek képek??
% TODO: Összegyűjteni az elérhető változatokat és kisebb összehasonlítást végezni.
QR kód:

A QR-kód (Quick Response-kód) egy kétdimenziós vonalkód (tulajdonképpen pontkód), amit a japán Denso-Wave cég fejlesztett ki 1994-ben. Nevét az angol Quick Response (=gyors válasz) rövidítéséből kapta, egyszerre utalva a gyors visszafejtési sebességre és a felhasználó által igényelt gyors reakcióra. 
Jó tulajdonsága, hogy bármilyen irányból készülhet róla fénykép vagy szkennelt kép, nem kell törődni a kód helyes tájolásával. Ez azért lehetséges, mert a kód megfejtésére, dekódolására szolgáló programok a három sarokban elhelyezett jellegzetes, minden QR-kódban azonos minta alapján el tudják dönteni, milyen irányban kell a kód pontjait értelmezni, feldolgozni, még akkor is, ha a kódbélyegről készült kép teljesen ferde. 
Másik jelentős pozitív tulajdonsága a kód skálázhatósága.
A QR kódót érdemes körülvenni egy fehér kerettel, hogy jól elkülüníthető legyen.
% https://hu.wikipedia.org/wiki/QR-k%C3%B3d

„Keretes” markerek:

Ezek a markerek két részből állnak; egy fekete keretből és egy belső mintából. A belső minta lehet kép, egyedi bináris minta, stb..

A keret fontos szerepet tölt ne, hisz egy fekete téglalapot/négyzetet viszonylag egyszerűen és gyorsan fel lehet ismerni egy képen, meghatározni annak pozícióját a kamerához képest, sarok pontjait, a középpontját.
Ilyen típusú marker az ARToolKit marker (középen egy kép van, ami bináris képként lesz kezelve), ARTag, AprilTag, ArUco. (mindháromnál egy bináris minta van középen)
Kombinálni is szokták az egyszerűbb markereket: a QR-kódot és a kerettel rendelkező markereket, vagy a QR kód van a keret belsejében vagy pedig a QR kód közepén szerepel például egy ArUco marker.

\SubSection{Betanítható markerek}

% Említés jelleggel, hogy milyen modernebb megoldások vannak hozzá.

A kiterjesztett valóság technikai fejlődéssel a markerek is bonyolultabbá váltak, lehetőség nyílt hétköznapi tárgyak, földrajzi koordináták (pl. PokemonGO), épületek és egyéb dolgok markerként kezelésére. Természetesen ezek detektálása nehezebb, illetve betanítást igényel.
Léteznek továbbá úgynevezett marker nélküli (markerless) AR alkalmazások is, a szenzorok felmérik környezetet és megfelelő helyre pozicionálják a megjelenítendő objektumot/objektumokat. 

\Section{Inerciális szenzorok}
Az inerciális szenzorok gyorsulásérzékelőkből, giroszkópokból állnak.
%KéSŐBB VISSZATÉRNI IDE 

% Mobil eszközök esetében megnézni, hogy milyen adatok állnak rendelkezésre.

% TODO: https://medium.com/@aliyasineser/aruco-marker-tracking-with-opencv-8cb844c26628

\Section{Az Aruco marker használata}
Az ArUco (Augmented Reality University of Cordoba) markereket 2014-ben fejlesztette ki kollégáival S.Garrido-Jurado Spanyolországban. 
Az ArUco könyvtár nyílt forráskódú és C++ nyelven íródott, OpenCV alapú. 
Ar ArUco markerek két részből állnak, egy fekete keretből és egy egyedi bináris mintából, ami azonosítja a markert.
Az OpenCV-s támogatottsága miatt került kiválasztásra a dolgozathoz. A detektálást számos beépített függvény segíti.
%[kép]
\SubSection{Típusai, használati módok}

Az ArUco markerek különböző könyvtárakban érhetők el, a bináris minta különbözteti meg őket. A bináris minta lehet 4x4 mérettől a 7x7 méretig elérhető és ezekhez, 50, 100, 250, 1000 ids. Azonban minél bonyolultabb egy minta, minél nagyobb annál nehezen felismerni, ezért én a  \texttt{DICT\_4X4\_50}-t használtam.

\SubSection{Kamera kalibrálása}
A kamera kalibrációjához 6x9-es sakktábla mintát használtam. (A külső sáv nem számolandó bele, az a felismerést könnyíti, a belső minta számít.) 
A kalibráció eredményességéhez a négyzeteknek szabályosnak, egyformának kell lenniük és fontos, hogy a nyomtatás során a minta ne torzuljon el, ne legyen átméretezve. 
A pontosság növelése érdekében a képeket érdemes minél változatosabb szögben és távolságból elkészíteni. Valamint ügyelni kell arra, hogy a lap egyenes legyen, ezért célszerű megfelelő támaszt használni hozzá.

A kalibráció során kapjuk meg a későbbiekben nélkülözhetetlen együttható (kamera) mátrixot, a disztorziós együtthatókat, az eltolási és forgatási vektorokat.

A kalibrációt végző függvénynek a sorok és oszlopok számára, a négyzetek oldalhosszára és a fent említett képekre van szüksége.
A folyamat során meg kell határozni a sakktábla minta sarokpontjait (findChessboardCorners), pontosabbá tenni azok koordinátáit (cornerSubPix). Majd ezen pontokat (projekciós pontjai a sakktáblába mintának) , a objektum pontjait ( a minta pontjai a minta terében), a kép méretét kell megadni a calibrateCamera() függvénynek. 
A kimenetei a függvénynek:
- cameraMatrix: $3 \times 3$ lebegőpontos kamera matrix
\[
A = 
\begin{bmatrix}
f_x & 0 & 0 \\
0 & f_y & 0 \\
0 & 0 & f_z \\
\end{bmatrix}
\]
- distortionCoefficient: Bemeneti/kimenti torzítási együttható vektora:
\[
(k_1, k_2, p_1, p_2, [, k_3 [, k_4, k_5, k_6, [, s_1, s_2, s_3, s_4 [, \tau_x, \tau_y]]]])
\]
4, 5, 8, 12 vagy 14 elemből állhat.
- rvecs: A mintákhoz tartozó elforgatási vektorok becsült kimeneti vektora.

- tvecs  A mintákhoz tartozó vektorok becsült kimeneti vektor


KÉP montázs
szöveg alá: A kalibráció felhasznált képek közül pár darab

%https://docs.opencv.org/master/d9/d0c/group__calib3d.html
%https://medium.com/@aliyasineser/opencv-camera-calibration-e9a48bdd1844

% TODO: Kalibrációs tutorial alapján lemérni a kamera paramétereit. Közben dokumentálni, hogy mi miért történt, és hogy milyen eredményeket adott.

\SubSection{Demo alkalmazás}

% TODO: Leírni, hogy hogyan lehet OpenCV-ből használni.
A markerek detektálását befolyásolja a fényviszony, a kamerához viszonyított pozíció, a minta bonyolultsága, a szög amiben látszik és a nagysága is. Továbbá óriási hátrány, hogy ha takarásra kerül egy része a markernek, akkor nem lesz felismerhető.

Az ArUco marker felismerésére használt demó alkalmazás bekalibrálja a kamerát (ha ez még nem történt meg), majd a kalibrációból kapott \texttt{camera\_matrix} és a \texttt{dist\_Coefficient} felhasználásával detektálja a markert az élő képen.

A folyamatos kapcsolat érdekében végtelen ciklust indul. Majd kapcsolatot kell teremteni a kamerával és mindig az adott pillanatnyi képpel dolgozni. 
Meg kell adni a megfelelő könyvtárat, amiben szerepel a marker, amit fel akarunk ismerni. Jelen esetben ez \texttt{DICT\_4X4\_100}.
\begin{python}
aruco_dict = aruco.Dictionary_get(aruco.DICT_4X4_100)
parameters = aruco.DetectorParameters_create()
parameters.adaptiveThreshConstant = 10
\end{python}

Majd ezeket és a kamera mátrixot és disztorziós együtthatókat felhasználva detektáljuk a markert.
Ha talál a képen markert, akkor körbe rajzolja azt, kijelöli a sarok pontját és ráteszi a marker koordináta rendszeréhez tartozó triédert.
