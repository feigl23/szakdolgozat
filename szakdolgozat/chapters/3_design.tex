\Chapter{Kamerapozíció becslése a virtuális térben}

\Section{Markerek}

\SubSection{Klasszikus markeres megoldások}

Vonalkód, QR-kód, Aruco marker

% TODO: Összegyűjteni az elérhető változatokat és kisebb összehasonlítást végezni.
QR kód:

A QR-kód (Quick Response-kód) egy kétdimenziós vonalkód (tulajdonképpen pontkód), amit a japán Denso-Wave cég fejlesztett ki 1994-ben. Nevét az angol Quick Response (=gyors válasz) rövidítéséből kapta, egyszerre utalva a gyors visszafejtési sebességre és a felhasználó által igényelt gyors reakcióra. 
Jó tulajdonsága, hogy bármilyen irányból készülhet róla fénykép vagy szkennelt kép, nem kell törődni a kód helyes tájolásával. Ez azért lehetséges, mert a kód megfejtésére, dekódolására szolgáló programok a három sarokban elhelyezett jellegzetes, minden QR-kódban azonos minta alapján el tudják dönteni, milyen irányban kell a kód pontjait értelmezni, feldolgozni, még akkor is, ha a kódbélyegről készült kép teljesen ferde. 
Másik jelentős pozitív tulajdonsága a kód skálázhatósága.
A QR kódót érdemes körülvenni egy fehér kerettel, hogy jól elkülüníthető legyen.


Fekete kerettel rendelkező markerek:

A marker két részből áll, van egy fekete keret és egy belső minta. A belső minta lehet kép, egyedi bináris minta, stb..
A lényeges a keret, hisz egy fekete téglalapot/négyzetet viszonylag egyszerűen és gyorsan fel lehet ismerni egy képen, meghatározni annak pozicióját a kamerához képest, sarok pontjait, a középpontját.
Ilyen marker az ARToolKit marker (középen egy kép van, ami bináris képként lesz kezelve), ARTag, AprilTag, ArUco. (mindháromnál egy bináris minta van középen)

Kombinálni is szokták a QR-kódot és a kerettel rendelkező markereket, vagy a QR kód van a keret belsejében vagy pedig a QR kód köpezén szerepel például egy ArUco marker.

Befolyásolja a markerek detektálását a fényviszony, a kamerához viszonyított pozíció, a minta bonolyultsága, a szög amiben látszik és a nagyság is. Továbbá hátrány, hogy ha takarásra kerül egy része a markernek, akkor nem kesz felismerhető.

\SubSection{Betanítható markerek}

% Említés jelleggel, hogy milyen modernebb megoldások vannak hozzá.

Azonban ma már bonyolultabb markerek is léteznek. Használhatók markerként megfelelő betanítás után objektumok(kólás üveg, pénz, stb..), GPS markerek (??).
Léteznek továbbá úgynevezett marker nélküli AR alkalmazások, a szenzorok felmérik környezetet és megfelelő helyre poziciónálják a megjelenítendő objektumot.

\Section{Inerciális szenzorok}

% Mobil eszközök esetében megnézni, hogy milyen adatok állnak rendelkezésre.

\Section{Az Aruco marker használata}

Az ArUco (Augmented Reality University of Cordoba) markereket 2014-ben fejlesztette ki kollégáival S.Garrido-Jurado. 
Az ArUco könyvtár nyílt forráskódú és C++ nyelven íródott. 
Ar ArUco markerek két részből állnak, egy fekete keretből és egy egyedi bináris mintából, ami azonosítja a markert.
Megfelelő kamera kalibráció után a marker kerete kerül felismerésre, meghatározódnak a sarokponjai, a középpontja és a poziciója a kamerához képest.

\SubSection{Típusai, használati módok}

Az ArUco markerek különböző könyvtárakban érhetők el, a bináris minta különbözteti meg őket. A bináris minta lehet 4x4 mérettől a 7x7 méretig elérhető és ezekhez, 50, 100, 250, 1000 ids. Azonban minél bonyolultabb egy minta, minél nagyobb annál nehezen felismerni, ezért én a  \texttt{DICT\_4X4\_50}-t használtam.

\SubSection{Kamera kalibrálása}

A kamera kalibrációjához sakktábla mintát használtam. Minden egyes négyzetnek egyforma méretűnk kell lennie a megfelelő ereményhez, jelen esetben ez 1.5 cm. 
A kallibrációhoz olyan képek szükségesek, ahol a minta különböző szögekből és távolságokból látható.

A calibrate() függvény végzni a kalibrációt a kódomban, ami a kép elérési útvonalát, a négyzetek méretét és az oszlopok és sorok számát várja el. A kalibráció során átadtam az elkészített képek elérési útvonalát, majd egy ciklusban minden képre végrehajtottam a megfelelő művelteket.
Első lépésként a kép fekete-fehér változatára van szükség, majd a beépített cv2.findChessboardCorners() meg kell kereni a sakktábla sarkait.
SUBPIX(???)

Az összegyűjtött pontok segítségével bekalibráljuk a kamerát (cv2.calibrateCamera()) és visszatérési értékként megkapjuk a következőket:
retval, cameraMatrix, distCoeffs, rvecs, tvecs. Ahol:

- cameraMatrix: Bemeneti/kimenti $3 \times 3$ lebegőpontos kamera matrix
\[
A = 
\begin{bmatrix}
f_x & 0 & 0 \\
0 & f_y & 0 \\
0 & 0 & f_z \\
\end{bmatrix}
\]
- distCoeffs: Bemeneti/kimenti torzítási együttható vektora:
\[
(k_1, k_2, p_1, p_2, [, k_3 [, k_4, k_5, k_6, [, s_1, s_2, s_3, s_4 [, \tau_x, \tau_y]]]])
\]
4, 5, 8, 12 vagy 14 elemből áll.
- rvecs: A mintákhoz tartozó elforgatási vektorok becsült kimeneti vektora(Rodrigues ), . Leírni a Rodriguest bővebben!!
- tvecs	 A mintákhoz tartozó vektorok becsült kimeneti vektor

% TODO: Kalibrációs tutorial alapján lemérni a kamera paramétereit. Közben dokumentálni, hogy mi miért történt, és hogy milyen eredményeket adott.

\SubSection{Demo alkalmazás}

% TODO: Leírni, hogy hogyan lehet OpenCV-ből használni.

A demó alkalmazás bekalibrálja a kamerát (ha ez még nem történt meg), majd a \texttt{coe\_mat} és a \texttt{dist\_mat} felhasználásával detektálja a markert az élő képen.
A folyamatos kapcsolat érdekében végtelen ciklust indítunk. 
Elsőnek is létre kell hozni a kapcsolatot a kamerával. A szürke változata kell az adott pillanatnyi képnek. Meg kell adni a megfelelő dictionary-t, amiben szerepel a marker, amit fel akarunk ismerni. Jelen esetben ez \texttt{DICT\_4X4\_50}.
\begin{python}
parameters = aruco.DetectorParameters_create()
parameters.adaptiveThreshConstant = 10
\end{python}
Majd ezeket és a mxt és dist felhasználva detektáljuk a markert.
Ha talál a képen markert, akkor körbe rajzolja a markert, kijelöli a sarok pontját és ráteszi a marker koordinátarendszeréhez tartozó triédert.
